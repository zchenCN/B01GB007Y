%%%%%%%%%%%%%%%%%%%%%%%%%%%%%%%%%%%%%%%%%%%%%%%%%%%%%%%%%%%%%%%%%%%%%%%%%%%%%%%%%%%%
% Do not alter this block (unless you're familiar with LaTeX
\documentclass{article}
\usepackage[margin=1in]{geometry} 
\usepackage{amsmath,amsthm,amssymb,amsfonts, fancyhdr, color, comment, graphicx, environ}
\usepackage{xcolor}
\usepackage{mdframed}
\usepackage[shortlabels]{enumitem}
\usepackage{indentfirst}
\usepackage{hyperref}
\usepackage[UTF8]{ctex}
\hypersetup{
    colorlinks=true,
    linkcolor=blue,
    filecolor=magenta,      
    urlcolor=blue,
}


\pagestyle{fancy}


\newenvironment{problem}[2][Problem]
    { \begin{mdframed}[backgroundcolor=gray!20] \textbf{#1 #2} \\}
    {  \end{mdframed}}

% Define solution environment
\newenvironment{solution}{\noindent \textbf{Solution}:}

%%%%%%%%%%%%%%%%%%%%%%%%%%%%%%%%%%%%%%%%%%%%%
%Fill in the appropriate information below
\lhead{}
\rhead{} 
\chead{\textbf{HOMEWORK09}}
%%%%%%%%%%%%%%%%%%%%%%%%%%%%%%%%%%%%%%%%%%%%%


\begin{document}

    \begin{problem}{1}
        计算下列积分
        \begin{itemize}
            \item[(1)] 
            $$\int_0^{2\pi}\frac{dx}{(a+b\cos x)^2}, \quad a > b >0$$
            \item[(2)] 
            $$\int_0^{\pi}\frac{d\theta}{1 + \sin^2\theta}$$
        \end{itemize}
    \end{problem}
    \begin{solution}
        \begin{itemize}
            \item[(1)] 令$z = e^{ix}$, 则$\cos x = \frac{z + z^{-1}}{2}$, $dx = \frac{dz}{iz}$
            \begin{equation*}\begin{split}
                I &= \int_{|z|=1}\frac{1}{\left(a + b\frac{z + z^{-1}}{2}\right)^2}\frac{dz}{iz}\\
                &= \frac{4}{ib^2}\int_{|z|=1}\frac{z}{\left(z^2 + \frac{2a}{b}z + 1\right)^2}\ dz\\
                &= \frac{4}{ib^2}\int_{|z|=1}\frac{z}{(z-\alpha)^2(z-\beta)^2}\ dz
            \end{split}\end{equation*}
            其中
            $$\alpha = \frac{-a + \sqrt{a^2-b^2}}{b}, \beta = \frac{-a - \sqrt{a^2 - b^2}}{b}$$
            由二次方程根与系数的关系可知, $|\alpha\beta| = 1$, 显然有$|\beta| > |\alpha|$, 所以$|\alpha| < 1$, 于是单位圆内被积函数只有$\alpha$一个二阶级点
            $$ Resf(\alpha)  =  -\frac{\alpha+\beta}{(\alpha-\beta)^3} = \frac{ab^3}{4b(a^2-b^2)^{3/2}}$$
            所以有留数定理可得
            $$I = \frac{2\pi a}{(a^2-b^2)^{3/2}}$$
            
            \item[(2)]显然
            $$ \int_0^{\pi}\frac{d\theta}{1 + \sin^2\theta} = \int_0^{\pi}\frac{2}{3 - \cos2\theta}\ d\theta$$
            令$z = e^{i2\theta}$, 则$\cos 2\theta = \frac{z + z^{-1}}{2}$, $d\theta = \frac{dz}{i2z}$
            \begin{equation*}\begin{split}
                I &= 2i\int_{|z|=1}\frac{1}{z^2 - 6z + 1}\ dz = \frac{i}{3}\\
                &= 2i \cdot 2\pi i Resf(3-2\sqrt{2})\\
                &= \frac{\pi}{\sqrt{2}}
            \end{split}\end{equation*}
        \end{itemize}
    \end{solution}
    
    \begin{problem}{2}
        计算下列积分
        \begin{itemize}
            \item[(1)] $$\int_0^{\infty}\frac{\cos x}{1 + x^4}\ dx$$
            \item[(2)] $$\int_{-\infty}^{\infty}\frac{x\sin x}{x^2 - 2x + 2}\ dx$$
        \end{itemize}
    \end{problem}
    \begin{solution}
        \begin{itemize}
            \item[(1)] 由偶函数的性质
            $$\int_0^{\infty}\frac{\cos x}{1 + x^4}\ dx = \frac{1}{2}\int_{-\infty}^{\infty}\frac{\cos x}{1 + x^4}\ dx$$
            我们先计算$f(z) = \frac{e^{iz}}{1 + z^4}$在以$R$为半径的上半圆周上(包含直径,$R$足够大, 记为$\Gamma$)积分的值, 
            $$ \int_{\Gamma}f(z)\ dz = \int_{C_R}f(z)\ dz + \int_{-R}^R\frac{\cos x}{1 + x^4}\ dx$$
            由若尔当引理, $$ \lim_{R\to\infty}f(z) dz = 0 \Rightarrow \int_{C_R}f(z) = 0$$
            由柯西留数定理
            \begin{equation*}
                    \int_{\Gamma}f(z)\ dz = 2\pi i \left(Resf(e^{i\pi/4}) + Resf(e^{i3\pi/4})\right) = \frac{\pi}{\sqrt{2}}e^{-\frac{1}{\sqrt{2}}}\left(\cos\frac{\sqrt{2}}{2} + \sin\frac{\sqrt{2}}{2}\right)
            \end{equation*}
            所以
            $$ I = \frac{1}{2}Re\{\int_{-\infty}^{\infty}f(z)dz \}= \frac{\pi}{2\sqrt{2}}e^{-\frac{1}{\sqrt{2}}}\left(\cos\frac{\sqrt{2}}{2} + \sin\frac{\sqrt{2}}{2}\right)$$
            \item[(2)]
            令
            $$ f(z) = \frac{ze^{iz}}{z^2 - 2z + 2} = \frac{ze^{iz}}{[z-(1+i)][z-(1-i)]}$$
            由大圆弧引理易知, 
            $$ \int_{C_R} f(z)dz = 0$$
            利用留数定理
            $$ \int_{\Gamma}f(z)dz = 2\pi iResf(1+i) = \pi(1+i)e^ie^{-1}$$
            所以
            $$ I = Img\{\int_{\Gamma}f(z)dz\} = \pi e^{-1}(\cos{1} + \sin{1})$$
        \end{itemize}
    \end{solution}

    \begin{problem}{3}
        计算下列积分
        \begin{itemize}
            \item[(1)]$$ \int_0^{\infty}\frac{x-\sin x}{x^3(1+x^2)}\ dx$$
            \item[(2)]$$ \int_0^{\infty}\frac{x^{\alpha-1}\ln x}{1 + x}\ dx, \quad 0 < \alpha < 1$$
        \end{itemize}
    \end{problem}
    \begin{solution}
        \begin{itemize}
            \item[(1)] 令$f(z) = \frac{1 + iz -e^{iz}}{z^3(1+z^2)}$ 
        \end{itemize}
    \end{solution}

\end{document}