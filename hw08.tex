%%%%%%%%%%%%%%%%%%%%%%%%%%%%%%%%%%%%%%%%%%%%%%%%%%%%%%%%%%%%%%%%%%%%%%%%%%%%%%%%%%%%
% Do not alter this block (unless you're familiar with LaTeX
\documentclass{article}
\usepackage[margin=1in]{geometry} 
\usepackage{amsmath,amsthm,amssymb,amsfonts, fancyhdr, color, comment, graphicx, environ}
\usepackage{xcolor}
\usepackage{mdframed}
\usepackage[shortlabels]{enumitem}
\usepackage{indentfirst}
\usepackage{hyperref}
\usepackage[UTF8]{ctex}
\hypersetup{
    colorlinks=true,
    linkcolor=blue,
    filecolor=magenta,      
    urlcolor=blue,
}


\pagestyle{fancy}


\newenvironment{problem}[2][Problem]
    { \begin{mdframed}[backgroundcolor=gray!20] \textbf{#1 #2} \\}
    {  \end{mdframed}}

% Define solution environment
\newenvironment{solution}{\noindent \textbf{Solution}:}

%%%%%%%%%%%%%%%%%%%%%%%%%%%%%%%%%%%%%%%%%%%%%
%Fill in the appropriate information below
\lhead{}
\rhead{} 
\chead{\textbf{HOMEWORK08}}
%%%%%%%%%%%%%%%%%%%%%%%%%%%%%%%%%%%%%%%%%%%%%


\begin{document}

    \begin{problem}{1}
        求下列函数在指定点~$z_0$~处的留数
        \begin{itemize}
            \item[(1)]~$\frac{e^{z^2}}{z-1}, z_0=1$~
            \item[(2)]~$\left(\frac{z}{1-\cos z}\right)^2, z_0=0$~
            \item[(3)]~$\frac{1}{z^2\sin z}, z_0=0$~
        \end{itemize}
    \end{problem}
    \begin{solution}
        \begin{itemize}
            \item[(1)] ~$z=1$~是~$f(z)$~的一阶级点, ~$Res f(1) = (z-1)\frac{e^{z^2}}{z-1}\bigg|_{z=1} = e$~
            \item[(2)] ~$z=0$~是~$f(z)$~的二阶级点, 
            \begin{align*}
                Res f(0) &= \left(\frac{z^4}{(1-\cos z)^2}\right)^{'}\bigg|_{z=0}\\
                &= \frac{4z^3(1-\cos z)-2z^4\sin z}{(1-\cos z)^3}\bigg|_{z=0}\\
                &=\frac{4z^3(\frac{z^2}{2}-\frac{z^4}{4!}+o(z^4)) -2z^4(z-\frac{z^3}{6}+o(z^3))}{\frac{z^6}{8}+o(z^6)}\bigg|_{z=0}\\
                &= 0
            \end{align*}
            \item[(3)]~$z=0$~是~$f(z)$~的三阶级点, 使用类似上小题对各个三角函数做Taylor展开的方法,可求得
            \begin{align*}
                Res f(0) &= \frac{1}{2}\left(\frac{z}{\sin z}\right)^{''}\bigg|_{z=0}\\
                &=\frac{z\sin^2 z- 2\sin z\cos z + 2z\cos^2
                z}{2\sin^3z}\bigg|_{z=0}\\
                &= \frac{1}{6}
            \end{align*}
        \end{itemize}
    \end{solution}

    \begin{problem}{2}
        求下列函数在复平面~$\mathbb{C}$~内每一个孤立奇点处的留数
        \begin{itemize}
            \item[(1)] ~$\frac{1}{z^3-z^5}$~
            \item[(2)] ~$\frac{1}{(z^2+1)^{m+1}}, m$~是正整数
            \item[(3)] ~$\frac{z}{1-\cos z}$~
            \item[(4)] ~$\cos\frac{1}{\sqrt{z}}$~
            \item[(5)] ~$\frac{1}{(z-1)Ln(z)}$~
        \end{itemize}
    \end{problem}
    \begin{solution}
        $\infty$不在复平面~$\mathbb{C}$~内,所以下面都不讨论$\infty$处的留数问题
        \begin{itemize}
            \item[(1)] 
            \begin{gather*}
                Res f(1) =   \frac{-1}{z^3(z+1)}\bigg|_{z=1} = -\frac{1}{2}\\
                Res f(-1) =   \frac{-1}{z^3(z-1)}\bigg|_{z=-1} =  -\frac{1}{2}\\
                Res f(0) = \frac{1}{2}\left(\frac{1}{1-z^2}\right)^{''}\bigg|_{z=0} = 1
            \end{gather*}
            \item[(2)]
                \begin{gather*}
                    Res f(i) = \frac{1}{m!}\frac{d^m}{d z^m}\left(\frac{1}{(z+i)^{m+1}}\right)\bigg|_{z=i} = (-1)^m\frac{(2m)!}{(m!)^2}\left(\frac{1}{(z+i)^{2m+1}}\right)\bigg|_{z=i} = -i\frac{(2m)!}{(m!)2^{2m+1}}\\
                    Res f(-i) = i\frac{(2m)!}{(m!)2^{2m+1}}
                \end{gather*}
            \item[(3)]
                \begin{gather*}
                    Res f(0) = \lim_{z\to 0}\left(\frac{z(z-n\pi)}{1-\cos z}\right) = 2\\
                    Res f(n\pi) = \lim_{z\to n\pi} \left(\frac{z(z-n\pi)}{1-\cos z}\right)^{'} = 2, n\neq 0
                \end{gather*}
            \item[(4)]
                \begin{equation*}
                    \cos \frac{1}{\sqrt{z}} = \sum_{n=0}^{\infty}(-1)^n\frac{1}{(2n)!}z^{-n} \Rightarrow Res f(0) = -\frac{1}{2}
                \end{equation*}
            \item[(5)]
                \begin{gather*}
                    Ln(1) = 0, Res f(1) = \lim_{z\to 1}\left(\frac{z-1}{Ln(z)}\right)^{'} = \frac{1}{2}\\
                    Ln(1) = i2k\pi, Res f(1) = \lim_{z\to 1} \left(\frac{1}{Ln(z)}\right) = \frac{1}{2k\pi i}
                \end{gather*}
        \end{itemize}
    \end{solution}

    \begin{problem}{3}
        求下列函数在无穷远点处的留数
        \begin{itemize}
            \item[(1)] ~$\frac{\cos z}{z}$~
            \item[(2)] ~$(z^2+1)e^z$~
            \item[(3)] ~$\sqrt{(z-1)(z-2)}$~
        \end{itemize}
    \end{problem}
    \begin{solution}
        \begin{itemize}
            \item[(1)] 
                \begin{equation*}
                    \frac{\cos z}{z} = \sum_{n=0}^{\infty}\frac{(-1)^nz^{2n-1}}{(2n!)}, Res f(\infty) = -c_{-1} = -1
                \end{equation*}
            
            \item[(2)]
                \begin{equation*}
                    (z^2+1)e^z = (z^2+1)\sum_{n=0}^{\infty}\frac{z^n}{n!}, Res f(\infty) = -c_{-1} = 0
                \end{equation*}
                
            \item[(3)]做代换$z^{'} = \frac{1}{z}$可求得$\infty$是$f(z)$的一阶极点,所以在无穷远处的Laurent展开只有一个正幂次项,即
            \begin{equation*}
                f(z) = \sqrt{(z-1)(z-2)} = c_{1}z + c_0 + c_{-1}z^{-1} + \cdots, 0 < R < |z| < \infty
            \end{equation*}
            则由$f(z)^2 = z^2 - 3z + 2$对比系数可得$c_{-1} = \pm\frac{1}{8}$, 所以$Resf(\infty)= \pm\frac{1}{8}$
        \end{itemize}
    \end{solution}
    
    \begin{problem}{4}
        计算积分
        \begin{itemize}
            \item[(1)] ~$\oint_{|z|=2}\frac{1}{z^3(z^{10} -2)}\ dz$~
            \item[(2)] ~$\int_0^{2\pi}\cos^{2n}\theta \ d\theta$~
        \end{itemize}
        \begin{itemize}
            \item[(1)]
            \begin{equation*}
                Res f(\infty) = -Res\{\left(\frac{t^{11}}{1-2t^{10}}\right), 0\} = 0 \Rightarrow I = 0
            \end{equation*}
            \item[(2)]
            \begin{align*}
                I &= \oint_{|z|=1}\left(\frac{z+z^{-1}}{2}\right)^{2n}\ \frac{dz}{iz}\\
                &= \oint_{|z|=1}\frac{1}{2^{2n}}\sum_{k=0}^{2n}\tbinom{2n}{k}z^kz^{k-2n}\frac{dz}{iz}\\
                &= \frac{2\pi}{2^{2n}}\tbinom{2n}{n}
            \end{align*}
        \end{itemize}
    \end{problem}
\end{document}