%%%%%%%%%%%%%%%%%%%%%%%%%%%%%%%%%%%%%%%%%%%%%%%%%%%%%%%%%%%%%%%%%%%%%%%%%%%%%%%%%%%%
% Do not alter this block (unless you're familiar with LaTeX
\documentclass{article}
\usepackage[margin=1in]{geometry} 
\usepackage{amsmath,amsthm,amssymb,amsfonts, fancyhdr, color, comment, graphicx, environ}
\usepackage{xcolor}
\usepackage{mdframed}
\usepackage[shortlabels]{enumitem}
\usepackage{indentfirst}
\usepackage{hyperref}
\usepackage[UTF8]{ctex}
\hypersetup{
    colorlinks=true,
    linkcolor=blue,
    filecolor=magenta,      
    urlcolor=blue,
}


\pagestyle{fancy}


\newenvironment{problem}[2][Problem]
    { \begin{mdframed}[backgroundcolor=gray!20] \textbf{#1 #2} \\}
    {  \end{mdframed}}

% Define solution environment
\newenvironment{solution}{\noindent \textbf{Solution}:}

%%%%%%%%%%%%%%%%%%%%%%%%%%%%%%%%%%%%%%%%%%%%%
%Fill in the appropriate information below
\lhead{}
\rhead{} 
\chead{\textbf{HOMEWORK15}}
%%%%%%%%%%%%%%%%%%%%%%%%%%%%%%%%%%%%%%%%%%%%%


\begin{document}

    \begin{problem}{1}
        求解格林函数的定解问题的解$G(x; x_0)$
        \begin{equation*}
            \begin{cases}
                    \frac{d^2G}{dx^2} = -\delta(x-x_0), \quad 0< x, x_0 < l\\
                    G(0, x_0) = G(l, x_0) = 0
            \end{cases}
        \end{equation*}
    \end{problem}
    \begin{solution}
        容易发现, 当$0 < x < x_0$和$x_0 < x < l$时, 都有$G^{''} = 0$, 利用边界条件, 容易有
        \begin{equation*}
        G(x, x_0) = 
            \begin{cases}
                ax, \quad 0 < x < x_0\\
                b(x-l), \quad x_0 < x < l
            \end{cases}
        \end{equation*}
        接下来利用$\delta$函数的选值的性质来确定待定常数, 对任意在$(0, l)$区间外为0的函数$f(x)$
        \begin{equation*}
            \begin{aligned}
                    f(x_0) & =\int_0^l\delta(x-x_0)f(x)dx  \\
                    &= \int_0^l-G^{''}f(x)dx\\
                    &= -\int_0^lG(x, x_0)f^{''}(x)dx\\
                    &= -\int_0^{x_0}axf^{''}(x)dx - \int_{x_0}^lb(x-l)f^{''}(x)dx\\
                    &= -\left(ax_0f^{'}(x_0) - af(x_0) + af(0)\right) - \left(-b(x_0-l)f^{'}(x_0) - bf(l) + bf(x_0)\right)\\
                    &= (a-b)f(x_0) - (a-b)x_0f^{'}(x_0) - blf^{'}(x_0)
            \end{aligned}
        \end{equation*}
        得到$$a = \frac{l-x_0}{l}, b=\frac{-x_0}{l}$$
        所以
         \begin{equation*}
        G(x, x_0) = 
            \begin{cases}
                \frac{(l-x_0)x}{l}, \quad 0 < x < x_0\\
                \frac{x_0(l-x)}{l}, \quad x_0 < x < l
            \end{cases}
        \end{equation*}      
    \end{solution}

    \begin{problem}{2}
        用格林函数法求解圆的狄利克莱问题
        \begin{equation*}
            \begin{cases}
                    \frac{\partial^2u}{\partial x^2} + \frac{\partial^2u}{\partial y^2} = -xy, \quad r < a\\
                    u|_{r=a} = 0
            \end{cases}
        \end{equation*}
    \end{problem}
    
    \begin{solution}
        有二维拉普拉斯方程的基本解
        \begin{equation*}
            \Gamma(r, x_0) = \frac{1}{2\pi}\ln{r}, \quad r = |x-x_0|
        \end{equation*}
        我们利用镜像法来求解二维拉普拉斯方程齐次狄利克雷边界条件的格林函数, 即
        \begin{equation*}
            \begin{aligned}
                \nabla^2 G(x, x_0) = \delta(x - x_0)\\
                G(x, x_0) = 0, \quad |x-x_0| = a
            \end{aligned}
        \end{equation*}
        我们在$x_0$的对称点$x^{*} = \frac{a^2 }{|x_0|}x_0$处放置一个汇,
        考虑问题
        $$ \nabla^2 u = \delta(x-x_0) - \delta(x - x^{*})$$
        显然当$|x| < a$时, $\nabla^2 u = \delta(x-x_0)$
        易知$$ u = \frac{1}{2\pi}ln{|x-x_0|} - \frac{1}{2\pi}ln{|x-x^{*}|} + c$$
        为满足边界条件
        $$c = \frac{1}{2\pi}ln{|x-x^{*}|} - \frac{1}{2\pi}ln{|x-x_0|} = \frac{1}{2\pi}\ln{\frac{a}{|x_0|}}$$
        其中
        $$ |x-x^{*}|^2 = a^2 + |x^{*}|^2 - 2x\cdot x^{*} = \frac{a^2}{|x_0|^2}\left(|x_0|^2 + a^2 - 2x\cdot x_0\right) = \frac{a^2}{|x_0|^2}|x-x_0|^2$$
        所以得到格林函数
        $$ G(x, x_0) = \frac{1}{2\pi}ln{|x-x_0|} - \frac{1}{2\pi}\ln{|x-x^{*}|} + \frac{1}{2\pi}\ln{\frac{a}{|x_0|}}$$
        利用格林表示公式
        $$ u(x, y) = \iint_{r<a}G(x, x_0)(-x_0y_0)dx_0dy_0$$
    \end{solution}
\end{document}
