%%%%%%%%%%%%%%%%%%%%%%%%%%%%%%%%%%%%%%%%%%%%%%%%%%%%%%%%%%%%%%%%%%%%%%%%%%%%%%%%%%%%
% Do not alter this block (unless you're familiar with LaTeX
\documentclass{article}
\usepackage[margin=1in]{geometry} 
\usepackage{amsmath,amsthm,amssymb,amsfonts, fancyhdr, color, comment, graphicx, environ}
\usepackage{xcolor}
\usepackage{mdframed}
\usepackage[shortlabels]{enumitem}
\usepackage{indentfirst}
\usepackage{hyperref}
\usepackage[UTF8]{ctex}
\hypersetup{
    colorlinks=true,
    linkcolor=blue,
    filecolor=magenta,      
    urlcolor=blue ,
}


\pagestyle{fancy}


\newenvironment{problem}[2][Problem]
    { \begin{mdframed}[backgroundcolor=gray!20] \textbf{#1 #2} \\}
    {  \end{mdframed}}

% Define solution environment
\newenvironment{solution}{\noindent \textbf{Solution}:}

%%%%%%%%%%%%%%%%%%%%%%%%%%%%%%%%%%%%%%%%%%%%%
%Fill in the appropriate information below
\lhead{}
\rhead{} 
\chead{\textbf{HOMEWORK03}}
%%%%%%%%%%%%%%%%%%%%%%%%%%%%%%%%%%%%%%%%%%%%%


\begin{document}

    \begin{problem}{1}
        求~$Ln(-i)$~和~$Ln(-3+4i)$~的值及主值.
    \end{problem}
    \begin{solution}
        由复数对数的定义~$Ln(z) = \ln{|z|} + i\arg(z)$~, 
        \begin{gather*}
            Ln(-i) = \ln1 + i(\frac{3\pi}{2} + 2k\pi) = i(\frac{3\pi}{2} + 2k\pi) \\
            Ln(-3+4i) = \ln5 + i(\arctan\frac{-4}{3} + 2k\pi) 
        \end{gather*}
        选取辐角在~$[0, 2\pi)$~的一支为主值,则有~$\ln(-i) = i\frac{3\pi}{2}$~, ~$\ln(-3+4i) = \ln5 + i\arctan\frac{-4}{3}$~
    \end{solution}
    
    \begin{problem}{2}
        判断下列函数是单值还是多值. 如果是多值,求其支点.
        \begin{itemize}
            \item [(1)] ~$z+\sqrt{z-1}$~          
            \item [(2)] ~$\frac{1}{1+Ln(z)}$~
            \item [(3)] ~$\frac{\sin\sqrt{z}}{\sqrt{z}}$~
            \item [(4)] ~$\sqrt{\frac{z-1}{z-2}}$~
            \item [(5)] ~$\sqrt[3]{(z-a)(z-b)}$~
            \item [(6)] ~$\sqrt{1-z^3}$~
        \end{itemize}
    \end{problem}
    \begin{solution}
        \begin{itemize}
            \item [(1)] ~$\sqrt{z-1}$~是多值函数,~$ 1, \infty$~是它的支点,所以~$z+\sqrt{z-1}$~也是多值函数,同样有支点~$1, \infty$~;
            \item [(2)] ~$Ln(z)$~是多值函数,~$ 0, \infty$~是它的支点,所以~$\frac{1}{1+Ln(z)}$~也是多值函数,同样有支点~$0, \infty$~;
            \item [(3)] 引起多值性的只能是~$\sqrt{z}$~,当~$z$~沿着绕~$z=0$~的一个足够小的闭合曲线一周回到初始点时,~$\sqrt{z}$~的辐角增加了~$\pi$~, 即从~$\sqrt{z}$~这一支变到了~$-\sqrt{z}$~这一支,但是~$\frac{\sin(-\sqrt{z})}{-\sqrt{z}} = \frac{\sin\sqrt{z}}{\sqrt{z}}$~,函数值没有发生变化,所以它是单值函数;
            \item [(4)]~$\sqrt{z-1}$~和~$\sqrt{z-2}$~都是多值函数,且支点分别为~$1, \infty$~和~$2, \infty$~, 所以它显然是多值函数且有支点~$1, 2$~. 同上小题一样分析,~$\infty$~不是它的支点;
            \item [(5)] 多值函数,支点有~$a, b, \infty$~;
            \item [(6)] 多值函数,支点有~$1, \frac{-1+\sqrt{3}i}{2}, \frac{-1-\sqrt{3}i}{2}, \infty$~
        \end{itemize}
    \end{solution}
    
    \begin{problem}{3}
        计算积分~$\int_0^{1+i}x-y+ix^2 \ dz$~,积分路径是直线段.
    \end{problem}
    \begin{solution}
        令~$\gamma(t) = t(1+i), 0\leq t\leq1$~,则有
        \begin{equation*}
            \int_0^{1+i}x-y+ix^2 \ dz=\int_0^1it^2(1+i)\ dt = \frac{i-1}{3}
        \end{equation*}
    \end{solution}
    
    \begin{problem}{4}
        利用积分不等式,证明:
        \begin{itemize}
            \item [(1)] ~$|I| = \left|\int_{-i}^ix^2 + iy^2\ dz\right| \leq 2,$~ 积分路径是直线段.
            \item [(2)] ~$|I| = \left|\int_{-i}^ix^2 + iy^2\ dz\right| \leq \pi,$~ 积分路径是连接~$-i$~到~$i$~的右半圆周.
        \end{itemize}
    \end{problem}
    \begin{solution}
        有不等式:
        \begin{equation*}
            \left|\int_Cf(z)\ dz\right| \leq Ml
        \end{equation*}
        其中~$M=\sup_{z\in C}|f(z)|$~, ~$l$~是~$C$~的长度.
        \begin{itemize}
            \item[(1)]~$M = sup_{z\in C}|x^2+iy| = \max_{y\in[-1, 1]}|y| = 1$~, ~$|I| \leq 2$~
            \item[(2)]~$M = sup_{z\in C}|x^2+iy| \leq |x^2|+|iy^2| = x^2 + y^2 = 1, $~, ~$|I| \leq 1 \times \pi = \pi $~
        \end{itemize}
    \end{solution}
    
    \begin{problem}{5}
        计算积分~$\int_{-1}^1|z| \ dz, $~积分路径是:
        \begin{itemize}
            \item [(1)] 直线段
            \item [(2)] 单位圆周的下半
        \end{itemize}
    \end{problem}
    \begin{solution}
        \begin{itemize}
            \item [(1)] ~$z = \gamma(t) = t, -1 \leq t\leq 1$~, 
            \begin{equation*}
                I = \int_{-1}^1|t|\ dt = 1
            \end{equation*}
            \item [(2)] ~$z = \gamma(\theta) = e^{i\theta}, -\pi \leq \theta\leq \pi$~,
            \begin{equation*}
                I = \int_{-\pi}^{0} d(e^{i\theta}) = e^{i\theta}\bigg|_{-\pi}^{0} = 2 
            \end{equation*}
        \end{itemize}
    \end{solution}
\end{document}