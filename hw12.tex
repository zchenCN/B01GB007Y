%%%%%%%%%%%%%%%%%%%%%%%%%%%%%%%%%%%%%%%%%%%%%%%%%%%%%%%%%%%%%%%%%%%%%%%%%%%%%%%%%%%%
% Do not alter this block (unless you're familiar with LaTeX
\documentclass{article}
\usepackage[margin=1in]{geometry} 
\usepackage{amsmath,amsthm,amssymb,amsfonts, fancyhdr, color, comment, graphicx, environ}
\usepackage{xcolor}
\usepackage{mdframed}
\usepackage[shortlabels]{enumitem}
\usepackage{indentfirst}
\usepackage{hyperref}
\usepackage[UTF8]{ctex}
\hypersetup{
    colorlinks=true,
    linkcolor=blue,
    filecolor=magenta,      
    urlcolor=blue,
}


\pagestyle{fancy}


\newenvironment{problem}[2][Problem]
    { \begin{mdframed}[backgroundcolor=gray!20] \textbf{#1 #2} \\}
    {  \end{mdframed}}

% Define solution environment
\newenvironment{solution}{\noindent \textbf{Solution}:}

%%%%%%%%%%%%%%%%%%%%%%%%%%%%%%%%%%%%%%%%%%%%%
%Fill in the appropriate information below
\lhead{}
\rhead{} 
\chead{\textbf{HOMEWORK12}}
%%%%%%%%%%%%%%%%%%%%%%%%%%%%%%%%%%%%%%%%%%%%%


\begin{document}

    \begin{problem}{1}
        求下列方程在$z=0$领域内的两个幂级数解
        \begin{itemize}
            \item[(1)] $(z^2-1)w^{''} + zw^{'} - w = 0$
            \item[(2)] $w^{''} - zw = 0$
        \end{itemize}
    \end{problem}
    \begin{solution}
        \begin{itemize}
            \item[(1)] 显然~$z=0$~是常点, 所以设方程的解~$w(z) = \sum\limits_{n=0}^{\infty}a_nz^n$~, 则
            $$w^{'} = \sum_{n=0}^{\infty}(n+1)a_{n+1}z^n, \quad w^{''} = \sum_{n=0}^{\infty}(n+1)(n+2)a_{n+2}z^n $$
            代入原方程可得
            $$ \sum_{n=2}^{\infty}\left[n(n-1)a_n-(n+1)(n+2)a_{n+2}\right]z^n-a_0-2a_2-6a_3z=0$$
            令$z$的各个次幂系数均为0, 得到
            $$ a_0+2a_2=0, a_3=0$$
            和递推公式
            $$ \frac{a_{n+2}}{a_n} = \frac{(n-1)(n+1)}{(n+1)(n+2)} = \frac{n-1}{n+2}$$
            所以容易有对大于4的偶数项有
            $$ a_{2n+2} = \frac{2n-1}{2n+2}a_{2n} = \frac{(2n-1)(2n-3)}{(2n+2)(2n)}a_{2n-2}= \frac{(2n-1)(2n-3)\cdots(1)}{(2n+2)(2n)\cdots(4)}a_{2n-2}=\cdots=\frac{2\cdot(2n-1)!!}{(2n+2)!!}a_2$$
            因为$a_3=0$, 所以对大于4的奇数项有$a_{2n+3}=0$.
            取$a_0=0, a_1=1$, 有$a_{2n+2}=0$, 所以
            $$ w = a_1z = z$$
            取$a_0=1, a_1=0$, 有
            $$a_2 = -\frac{1}{2}, a_{2n+2}=-\frac{\cdot(2n-1)!!}{(2n+2)!!}$$
            所以
            $$ w = 1 + \sum_{n=1}^{\infty}a_nz^n = 1 - \frac{\cdot(2n-3)!!}{(2n)!!}z^n$$
            \item[(2)]
            显然~$z=0$~是常点, 所以设方程的解~$w(z) = \sum\limits_{n=0}^{\infty}a_nz^n$~, 则
            $$w^{'} = \sum_{n=0}^{\infty}(n+1)a_{n+1}z^n, \quad w^{''} = \sum_{n=0}^{\infty}(n+1)(n+2)a_{n+2}z^n $$
            代入原方程可得
            $$ 2a_2 + \sum_{n=1}^{\infty}[(n+1)(n+2)a_{n+2}-a_{n-1}]z^n = 0$$
            得到
            $$ a_2 = 0, \frac{a_{n+2}}{a_{n-1}} = \frac{1}{(n+1)(n+2)}, n\geq 1$$
            所以利用递推关系
            \begin{gather*}
                a_{3n} = \frac{a_{3(n-1)}}{(3n-1)(3n)} = \cdots = \frac{a_0\Gamma(\frac{2}{3})}{3^{2n}\Gamma(n+1)\Gamma(n+\frac{2}{3})}\\
                a_{3n+1} = \frac{a_1\Gamma(\frac{4}{3})}{3^{2n}\Gamma(n+1)\Gamma(n+\frac{4}{3})}\\
                a_{3n+2} = 0
            \end{gather*}
            取$a_0 = 0, a_1 = 1$则
            $$ w = \sum_{n=0}^{\infty}\frac{a_1\Gamma(\frac{4}{3})}{3^{2n}\Gamma(n+1)\Gamma(n+\frac{4}{3})}z^{3n+1}$$
            取$a_0 = 1, a_1 = 0$则
            $$ w = \sum_{n=0}^{\infty}\frac{a_0\Gamma(\frac{2}{3})}{3^{2n}\Gamma(n+1)\Gamma(n+\frac{2}{3})}z^{3n}$$
        \end{itemize}
    \end{solution}
    
    \begin{problem}{2}
        求下列方程在$z=0$领域内的两个幂级数解
        \begin{itemize}
            \item[(1)] $zw^{''} - zw^{'} + w = 0$
            \item[(2)] $zw^{''} + (z-1)w^{'} + w = 0$
        \end{itemize}
    \end{problem}
    \begin{solution}
        \begin{itemize}
            \item[(1)]显然~$z=0$~是正则奇点, 所以设方程的解~$w(z) = \sum\limits_{n=0}^{\infty}a_nz^{n+\rho}$~, 则
            $$w^{'} = \sum_{n=0}^{\infty}a_n(n+\rho)z^{n+\rho-1}, \quad w^{''} = \sum_{n=0}^{\infty}a_n(n+\rho)(n+\rho-1)z^{n+\rho-2} $$
            代入原方程得
            $$ a_0\rho(\rho-1) + \sum_{n=0}^{\infty}[a_{n+1}(n+\rho)(n+\rho+1)-a_n(n+\rho-1)] = 0$$
            得到指标方程
            $$ \rho(\rho-1) = 0$$
            和递推关系
            $$ a_{n+1}(n+\rho)(n+\rho+1)-a_n(n+\rho-1) = 0, n\geq 0$$
            当$\rho=1$时, 利用递推关系
            $$ a_{n+1} = \frac{n}{(n+1)(n+2)}a_n$$
            可得$a_n = 0, n\geq 1$
            取$a_0 = 1$得到方程的一个解
            $$ w_1(z) = z$$
            则
            \begin{equation*}
                \begin{split}
                    w_2(z) &= w_1(z)\int^z\frac{1}{s^2}e^s\ ds\\
                    &= z\int^z\sum_{n=0}^{\infty}\frac{1}{n!}s^{n-2}\ ds\\
                    &= z\int^z \frac{1}{s^2} + \frac{1}{s} + \sum_{n=0}^{\infty}\frac{1}{(n+2)!}s^n\ ds\\
                    &= z\left(-\frac{1}{z} + \ln z + \sum_{n=0}^{\infty}\frac{1}{(n+2)!(n+1)}z^{n+1}\right)\\
                    &= -1 + z\ln z + \sum_{n=0}^{\infty}\frac{1}{n!(n-1)}z^n
                \end{split}
            \end{equation*}
            \item[(2)]显然~$z=0$~是正则奇点, 所以设方程的解~$w(z) = \sum\limits_{n=0}^{\infty}a_nz^{n+\rho}$~, 则
            $$w^{'} = \sum_{n=0}^{\infty}a_n(n+\rho)z^{n+\rho-1}, \quad w^{''} = \sum_{n=0}^{\infty}a_n(n+\rho)(n+\rho-1)z^{n+\rho-2} $$
            代入原方程得
            $$ a_0\rho(\rho-2)z^{-1+\rho} + \sum_{n=0}^{\infty}[a_{n+1}(n+\rho+1)(n+\rho-1) + a_n(n+\rho+1)]z^{n+\rho}=0$$
            得到指标方程
            $$\rho(\rho-2)z^{-1+\rho} = 0 $$
            和递推关系
            $$ a_{n+1}(n+\rho-1) + a_n = 0, n\geq 0$$
            当$\rho=2$时, 
            $$a_{n+1} = -\frac{1}{n+1}a_n= \cdots = (-1)^{n+1}\frac{1}{(n+1)!}a_0$$
            取$a_0 = 1$得
            $$ a_n = (-1)^n\frac{1}{n!}$$
            所以
            $$ w_1(z) = z^2\sum_{n=0}^{\infty} = z^2e^{-z}$$
            同样利用上题积分的方法, 可得
            $$ w_2(z) = z^2e^{-z}\left(-\frac{1}{2z^2}-\frac{1}{z} + \frac{1}{2}\ln z + \sum_{n=0}^{\infty}\frac{1}{(n+1)(n+3)!}z^n\right)$$
        \end{itemize}
    \end{solution}

    \begin{problem}{3}
        求方程
        $$ \frac{d^2u}{dz^2} + \frac{2}{z}\frac{du}{dz} + m^2u = 0$$
        在$z=0$领域内的两个线性无关的.                                                       
    \end{problem}
    \begin{solution}
        $z=0$是正则奇点, 设$u = \sum_{n=0}^{\infty}a_nz^{n+\rho}$
        代入原方程得到指标方程
        $$ \rho(\rho+1)=0$$
        和递推关系
        $$ a_1(\rho+1)(\rho+2) = 0, \quad a_{n+2}(n+\rho+2)(n+\rho+3) + m^2a_n = 0, n\geq 0$$
        当$\rho=0$时, 得到$a_1 = 0$和$a_{n+2} = -\frac{m^2}{(n+3)(n+2)}a_n$, 所以有
        $$ a_{2n} = (-1)^n\frac{m^{2n}}{(2n+1)!}a_0, \quad a_{2n+1}=0$$
        取$a_0 = 1$得到
        $$ u(z) = \frac{\sin(mz)}{mz}, m\neq 0$$
        当$\rho=-1$时, 
        $$a_{n+2} = -\frac{m^2}{(n+1)(n+2)}a_n$$
        所以有
        $$ a_{2n} = -\frac{m^2}{(2n)!}a_0, \quad a_{2n+1} = -\frac{m^2}{(2n+1)!}a_1$$
        所以
        $$ u(z) = a_0\frac{\cos(mz)}{z} + a_1\frac{\sin(mz)}{mz}$$
    \end{solution}

\end{document}