%%%%%%%%%%%%%%%%%%%%%%%%%%%%%%%%%%%%%%%%%%%%%%%%%%%%%%%%%%%%%%%%%%%%%%%%%%%%%%%%%%%%
% Do not alter this block (unless you're familiar with LaTeX
\documentclass{article}
\usepackage[margin=1in]{geometry} 
\usepackage{amsmath,amsthm,amssymb,amsfonts, fancyhdr, color, comment, graphicx, environ}
\usepackage{xcolor}
\usepackage{mdframed}
\usepackage[shortlabels]{enumitem}
\usepackage{indentfirst}
\usepackage{hyperref}
\usepackage[UTF8]{ctex}
\hypersetup{
    colorlinks=true,
    linkcolor=blue,
    filecolor=magenta,      
    urlcolor=blue,
}


\pagestyle{fancy}


\newenvironment{problem}[2][Problem]
    { \begin{mdframed}[backgroundcolor=gray!20] \textbf{#1 #2} \\}
    {  \end{mdframed}}

% Define solution environment
\newenvironment{solution}{\noindent \textbf{Solution}:}

%%%%%%%%%%%%%%%%%%%%%%%%%%%%%%%%%%%%%%%%%%%%%
%Fill in the appropriate information below
\lhead{}
\rhead{} 
\chead{\textbf{HOMEWORK04}}
%%%%%%%%%%%%%%%%%%%%%%%%%%%%%%%%%%%%%%%%%%%%%


\begin{document}

    \begin{problem}{1}
        计算积分~$\oint_c\frac{\bar{z}}{|z|} \ dz,$~ 其中~$c$~为正向圆周
        \begin{itemize}
            \item [(1)] ~$|z| = 3$~
            \item [(2)] ~$|z| = 4$~
        \end{itemize}
    \end{problem}
    \begin{solution}
        \begin{equation*}
            \oint_{|z|=r}\frac{\bar{z}}{|z|} \ dz = \int_0^{2\pi}\frac{re^{-i\theta}}{r}ire^{i\theta}\ d\theta = ir\int_0^{2\pi}d\theta = i2\pi r
        \end{equation*}
    \end{solution}

    \begin{problem}{2}
        计算~$\oint_c\frac{1}{z^2-z} \ dz,$~ 其中~$c$~为包含圆周~$|z|=1$~在内的任何正向简单闭曲线.
    \end{problem}
    \begin{solution}
        容易证明~$\int_{|z-a|=r}\frac{dz}{z-a} = 2\pi i$~
        由柯西定理可知:
        \begin{equation*}
            \oint_c\frac{1}{z^2-z} \ dz = \oint_{|z-1|=r}\frac{1}{z^2-z} \ dz = \oint_{|z-1|=r}\frac{1}{z-1} - \oint_{|z-1|=r}\frac{1}{z} \ dz =2\pi i -2\pi i = 0
        \end{equation*}
    \end{solution}

    \begin{problem}{3}
        计算下列积分:
        \begin{itemize}
            \item [(1)] ~$\oint_{|z|=3}\frac{dz}{z^2 + 2z + 3}$~
            \item [(2)] ~$\oint_{|z|=1}\frac{dz}{\cos z}$~
        \end{itemize}
    \end{problem}
    \begin{solution}
        \begin{itemize}
            \item[(1)] 假设~$z^2+2z+3 = (z-z_1)(z-z_2)$~
            \begin{equation*}
                I = \oint_{|z|=3}\frac{dz}{(z-z_1)(z-z_2)} = \frac{1}{z_1-z_2}\oint_{|z|=3}\frac{1}{z-z_1}-\frac{1}{z-z_2}\ dz =0
            \end{equation*}
            \item[(2)]由柯西定理可知为0
        \end{itemize}
    \end{solution}
    
    \begin{problem}{4}
        计算~$\oint_l\frac{dz}{(z-a)(z-b)} \ dz,$~ ~$l$~是包围~$a, b$~两点的圆周
    \end{problem}
    \begin{solution}
        同上题
    \end{solution}
    
    \begin{problem}{5}
        计算:
        \begin{itemize}
            \item [(1)] ~$\int_{-2}^{-2+i}(z+2)^2\ dz$~
            \item [(2)] ~$\int_0^{\pi+2i}\cos\frac{z}{2}\ dz$~
        \end{itemize}
    \end{problem}
    \begin{solution}
        \begin{itemize}
            \item [(1)] ~$\int_{-2}^{-2+i}(z+2)^2\ dz = \frac{(z+2)^3}{3}\bigg|_{-2}^{-2+i} = \frac{-3i}{3}$~
            \item [(2)] ~$\int_0^{\pi+2i}\cos\frac{z}{2}\ dz = 2\sin\frac{z}{2}\bigg|_0^{\pi+2i} = 2\sin(\frac{\pi}{2}+i) = e+e^{-1}$~
        \end{itemize}
    \end{solution}
\end{document}