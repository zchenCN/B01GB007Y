%%%%%%%%%%%%%%%%%%%%%%%%%%%%%%%%%%%%%%%%%%%%%%%%%%%%%%%%%%%%%%%%%%%%%%%%%%%%%%%%%%%%
% Do not alter this block (unless you're familiar with LaTeX
\documentclass{article}
\usepackage[margin=1in]{geometry} 
\usepackage{amsmath,amsthm,amssymb,amsfonts, fancyhdr, color, comment, graphicx, environ}
\usepackage{xcolor}
\usepackage{mdframed}
\usepackage[shortlabels]{enumitem}
\usepackage{indentfirst}
\usepackage{hyperref}
\usepackage[UTF8]{ctex}
\hypersetup{
    colorlinks=true,
    linkcolor=blue,
    filecolor=magenta,      
    urlcolor=blue,
}


\pagestyle{fancy}


\newenvironment{problem}[2][Problem]
    { \begin{mdframed}[backgroundcolor=gray!20] \textbf{#1 #2} \\}
    {  \end{mdframed}}

% Define solution environment
\newenvironment{solution}{\noindent \textbf{Solution}:}

%%%%%%%%%%%%%%%%%%%%%%%%%%%%%%%%%%%%%%%%%%%%%
%Fill in the appropriate information below
\lhead{}
\rhead{} 
\chead{\textbf{HOMEWORK16}}
%%%%%%%%%%%%%%%%%%%%%%%%%%%%%%%%%%%%%%%%%%%%%


\begin{document}
    
    \begin{problem}{1}
        用格林函数法求解
        \begin{equation*}
            \begin{cases}
                    \frac{\partial u}{\partial t} - a^2\frac{\partial^2 u}{\partial x^2} = A\sin{wt}, \quad 0 < x < l, t>0\\
                    \frac{\partial u}{\partial x}(0, t) = \frac{\partial u}{\partial x}(l, t) = 0\\
                    u(x, 0) = 0
            \end{cases}
        \end{equation*}
    \end{problem}
    \begin{solution}
        设格林函数$G(x, t, \xi, \tau)$满足方程
        \begin{equation*}
            \begin{cases}
                    \frac{\partial G}{\partial t} - a^2\frac{\partial^2 G}{\partial x^2} = \delta(x-\xi)\delta(t-\tau), \quad 0 < x < l, t>0\\
                    \frac{\partial G}{\partial x}(0, t) = \frac{\partial G}{\partial x}(l, t) = 0\\
                    G(x, 0) = 0
            \end{cases}
        \end{equation*}
        容易验证
        $$ u(x, t) = \int_0^{\infty}\int_{-\infty}^{\infty}A\sin{w\tau}G(x, t, \xi, \tau)d\xi d\tau$$
        是原方程的解。下面我们使用本征函数法求解上述关于格林函数的方程。边界条件对应的本征函数集为
        $$ X_n(x) = A_n\cos\frac{n\pi}{l}x, n=0, 1, 2, \cdots$$
        设方程解有如下形式
        $$ G(x, t) = \sum_{n=0}^{\infty}g_n(t)\cos\frac{n\pi}{l}x$$
        由初始条件可得
        $$ g_n(0) = 0, n=0, 1, 2, \cdots$$
        将上式代入方程易得
        $$ \sum_{n=0}^{\infty}\left[g_n^{'}(t) + \left(\frac{an\pi}{l}\right)^2g_n(t)\right]\cos{\frac{n\pi}{l}x} = \delta(x-\xi)\delta(t-\tau)$$
        设
        $$ \delta(x-\xi)\delta(t-\tau) = \sum_{n=0}^{\infty}f_n(t)\cos{\frac{n\pi}{l}x}$$
        容易得到
        $$ f_n(t) = \frac{2}{l}\int_0^l\delta(x-\xi)\delta(t-\tau)\cos{\frac{n\pi}{l}x}dx = \frac{2}{l}\cos{\frac{n\pi\xi}{l}}\delta(t-\tau)$$
        接下来需要求解定解问题
        \begin{equation*}
            \begin{cases}
                    g_n^{'}(t) + \left(\frac{an\pi}{l}\right)^2g_n(t) = f_n(t)\\
                    g_n(0) = 0
            \end{cases}
        \end{equation*}
        有拉普拉斯变换及其性质易得
        $$ pG_n(p) + \left(\frac{an\pi}{l}\right)^2G_n(p) = F_n(p)$$
        解得
        $$ G_n(p) = \frac{F_n(p)}{p + \left(\frac{an\pi}{l}\right)^2}$$
        由拉普拉斯变换卷积的性质
        $$ g_n(t) = \int_0^texp\left\{-\left(\frac{an\pi}{l}\right)^2(t-s)\right\}f_n(s)ds = \frac{2}{l}\cos{\frac{n\pi\xi}{l}}exp\left\{-\left(\frac{an\pi}{l}\right)^2(t-\tau)\right\}H(t-\tau)$$
        其中$H$是Heaviside函数。
    \end{solution}
    
    \begin{problem}{2}
        用格林函数法求解
        \begin{equation*}
            \begin{cases}
                    \frac{\partial^2 u}{\partial t^2} - a^2\frac{\partial^2 u}{\partial x^2} = A\cos{\frac{x}{l}}\sin{wt}, \quad 0 < x < l, t>0\\
                    \frac{\partial u}{\partial x}(0, t) = \frac{\partial u}{\partial x}(l, t) = 0\\
                    u(x, 0) = \frac{\partial u}{\partial t}(x, 0) = 0
            \end{cases}
        \end{equation*}
    \end{problem}
    \begin{solution}
        设格林函数$G(x, t, \xi, \tau)$满足方程
        \begin{equation*}
            \begin{cases}
                    \frac{\partial^2 G}{\partial t^2} - a^2\frac{\partial^2 G}{\partial x^2} = \delta(x-\xi)\delta(t-\tau), \quad 0 < x < l, t>0\\
                    \frac{\partial G}{\partial x}(0, t) = \frac{\partial G}{\partial x}(l, t) = 0\\
                    G(x, 0) = \frac{\partial G}{\partial t}(x, 0) = 0
            \end{cases}
        \end{equation*}
        容易验证
        $$ u(x, t) = \int_0^{\infty}\int_{-\infty}^{\infty}A\cos{\frac{\xi}{l}}\sin{w\tau}G(x, t, \xi, \tau)d\xi d\tau$$
        是原方程的解。同样地,我们使用本征函数法求解上述格林函数的方程,推到类似上题,可得
        $$ \sum_{n=0}^{\infty}\left[g_n^{''}(t) + \left(\frac{an\pi}{l}\right)^2g_n(t)\right]\cos{\frac{n\pi}{l}x} = \delta(x-\xi)\delta(t-\tau)$$
        且
        $$ g_n(0) = g^{'}_n(0) = 0$$
        同样的使用拉普拉斯变换得
        $$ G_n(p) = \frac{F_n(p)}{p^2 + \left(\frac{an\pi}{l}\right)^2}$$
        所以
        $$ g_n(t) = \frac{2}{na\pi}\cos{\frac{n\pi\xi}{l}}\sin{\frac{na\pi}{l}(t-\tau)}H(t-\tau)$$
    \end{solution}


\end{document}