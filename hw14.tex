%%%%%%%%%%%%%%%%%%%%%%%%%%%%%%%%%%%%%%%%%%%%%%%%%%%%%%%%%%%%%%%%%%%%%%%%%%%%%%%%%%%%
% Do not alter this block (unless you're familiar with LaTeX
\documentclass{article}
\usepackage[margin=1in]{geometry} 
\usepackage{amsmath,amsthm,amssymb,amsfonts, fancyhdr, color, comment, graphicx, environ}
\usepackage{xcolor}
\usepackage{mdframed}
\usepackage[shortlabels]{enumitem}
\usepackage{indentfirst}
\usepackage{hyperref}
\usepackage[UTF8]{ctex}
\hypersetup{
    colorlinks=true,
    linkcolor=blue,
    filecolor=magenta,      
    urlcolor=blue,
}


\pagestyle{fancy}


\newenvironment{problem}[2][Problem]
    { \begin{mdframed}[backgroundcolor=gray!20] \textbf{#1 #2} \\}
    {  \end{mdframed}}

% Define solution environment
\newenvironment{solution}{\noindent \textbf{Solution}:}

%%%%%%%%%%%%%%%%%%%%%%%%%%%%%%%%%%%%%%%%%%%%%
%Fill in the appropriate information below
\lhead{}
\rhead{} 
\chead{\textbf{HOMEWORK14}}
%%%%%%%%%%%%%%%%%%%%%%%%%%%%%%%%%%%%%%%%%%%%%


\begin{document}

    \begin{problem}{1}
        求解下列定界问题
        \begin{equation*}
            \begin{cases}
                    \frac{\partial^2u}{\partial t^2} = a^2 \frac{\partial^2u}{\partial x^2},\quad 0 < x < l, t > 0\\
                    u(0, t) = \cos\frac{a\pi}{l}t, \quad \frac{\partial  u}{\partial x}(l, t) = 0\\
                    u(x, 0) = \cos\frac{\pi}{l}x,\quad  \frac{\partial }{\partial t}u(x, 0) = \sin\frac{\pi}{2 l}x
            \end{cases}+
        \end{equation*}
    \end{problem}
    \begin{solution}
    % \begin{itemize}
    % \item[方法1:]
    我们希望同时将方程和边界条件齐次化令$u(x, t) = w(x, t) + v(t, x)$, 设
    $$ v(x, t) = f(x)\cos{\frac{a\pi}{l}t}$$
    则有
    \begin{equation*}
        \begin{aligned}
            \frac{\partial^2 w}{\partial t^2} & = \frac{\partial^2 u}{\partial t^2} - \frac{\partial^2 v}{\partial t^2}\\
            &= a^2\frac{\partial^2 u}{\partial x^2} + f(x)\left(\frac{a\pi}{l}\right)^2\cos{\frac{a\pi}{l}t}\\
        \end{aligned}
    \end{equation*}
    对比$$\frac{\partial^2 w}{\partial x^2} = \frac{\partial^2 u}{\partial x^2} - f^{''}(x)\cos{\frac{a\pi}{l}t}$$
    令
    $$ f^{''}(x) + \left(\frac{\pi}{l}\right)^2f(x) = 0$$
    再由边界条件
    $$ w(0, t) = u(0, t) - v(0, t) = \cos{\frac{a\pi}{l}t} - f(0)\cos{\frac{a\pi}{l}t}= 0$$
    和
    $$ \frac{\partial}{\partial x}w(l, t) = \frac{\partial}{\partial x}u(l, t) - f^{'}(l)\cos{\frac{a\pi}{l}t} = 0$$
    令$$ f(0) = 1, \quad f^{'}(l) = 0$$ 
    求解关于$f(x)$的常微分方程定解问题
    \begin{equation*}
        \begin{cases}
                f^{''}(x) + \left(\frac{\pi}{l}\right)^2f(x) = 0\\
                f(0) = 1, \quad f^{'}(l) = 0
        \end{cases}
    \end{equation*}
    容易解得
    $$ f(x) = \cos{\frac{\pi}{l}x}$$
    那么容易得到关于$w(x, t )$的齐次方程
    \begin{equation*}
        \begin{cases}
               \frac{\partial^2w}{\partial t^2} =   a^2\frac{\partial^2w}{\partial x^2} \\
                 w(0, t) = 0, \quad \frac{\partial}{\partial x}w(l, t) = 0\\
                 w(x, 0) = 0, \quad \frac{\partial}{\partial t}w(x, 0) = \sin\frac{\pi}{2 l}x
        \end{cases}
    \end{equation*}
    利用本征值方法求解该方程,由边界条件可知对应本征函数集合为
    $$ \sin\frac{(2n+1)\pi}{2l}x, \quad n= 0, 1, \cdots$$
    设方程解有级数形式
    $$ w(x, t) = \sum_{n\geq0}g_n(t)\sin\frac{(2n+1)\pi}{2l}x$$
    其中$g_n(t)$应该满足
    $$ g_n(t) = \frac{2}{l}\int_0^lw(x, t)\sin\frac{(2n+1)\pi}{2l}xdx$$
    那么
    \begin{gather*}
        g_n(0) = \frac{2}{l}\int_0^lw(x, 0)\sin\frac{(2n+1)\pi}{2l}xdx = 0\\
        g_n^{'}(0) = \frac{2}{l}\int_0^l\sin{\frac{\pi}{2l}x}\sin\frac{(2n+1)\pi}{2l}xdx = \delta_{0, n}
    \end{gather*}
    代入方程得到
    $$ \sum_{n\geq0}\left[g_n^{''}(t) + \left(\frac{(2n+1)a\pi}{2l}\right)^2\right]\sin\frac{(2n+1)\pi}{2l}x = 0$$
    可得关于$g_n(t)$的常微分方程定界问题
    \begin{equation*}
        \begin{cases}
            g_n^{''}(t) + \left(\frac{(2n+1)a\pi}{2l}\right)^2g_n(t) = 0\\
            g_n(0) = 0, \quad g_n^{'}(l) = \delta_{0, n}
        \end{cases}
    \end{equation*}
    容易解得
    \begin{equation*}
        g_0(t) = \frac{2l}{a\pi}\sin{\frac{a\pi}{2l}t}\\
        g_n(t) = 0, \quad n=1, 2, \cdots
    \end{equation*}
    所以可得
    $$ u(x, t) = w(x, t) + v(x, t) = \frac{2l}{a\pi}\sin{\frac{a\pi}{2l}t}\sin{\frac{\pi}{2l}x} + \cos{\frac{\pi}{l}x}\cos{\frac{a\pi}{l}t}$$
    \end{solution}
    % \item[方法2]
    %     令$v(x, t) = \cos\frac{a\pi}{l}t$, $w(x, t) = u(x, t) - v(x, t)$, 则容易验证$w$满足齐次边界条件
    %     $$ w(0, t) = u(0, t) - v(0, t) = \cos\frac{a\pi}{l}t - \cos\frac{a\pi}{l}t = 0$$
    %     和
    %     $$ \frac{\partial}{\partial x}w(l, t) = \frac{\partial}{\partial x}u(l, t) - \frac{\partial}{\partial x}v(l, t) = 0 $$
    %     同样的可以验证初始条件,
    %     $$ w(x, 0) = u(x, 0) - v(x, 0) = \cos\frac{\pi}{l}x - 1$$
    %     和
    %     $$ \frac{\partial}{\partial t}w(x, 0) = \frac{\partial}{\partial t}u(x, 0) - \frac{\partial}{\partial t}v(x, 0) = \sin\frac{\pi}{2l}x$$
    %     代入方程
    %     \begin{equation*}
    %         \begin{aligned}
    %          \frac{\partial^2w}{\partial t^2} &= \frac{\partial^2u}{\partial t^2} - \frac{\partial^2v}{\partial t^2}\\
    %          &= a^2\frac{\partial^2u}{\partial x^2} + \left(\frac{a\pi}{l}\right)^2\cos\frac{a\pi}{l}t\\
    %          &= a^2\frac{\partial^2w}{\partial x^2} + \left(\frac{a\pi}{l}\right)^2\cos\frac{a\pi}{l}t
    %         \end{aligned}
    %     \end{equation*}
    %     接下来我们使用本征函数法求解关于$w$的非齐次定界问题
    %     \begin{equation*}
    %         \begin{cases}
    %              \frac{\partial^2w}{\partial t^2} =   a^2\frac{\partial^2w}{\partial x^2} + \left(\frac{a\pi}{l}\right)^2\cos\frac{a\pi}{l}t\\
    %              w(0, t) = 0, \quad \frac{\partial}{\partial x}w(l, t) = 0\\
    %              w(x, 0) = \cos\frac{\pi}{l}x - 1, \quad \frac{\partial}{\partial t}w(x, 0) = \sin\frac{\pi}{2l}x
    %         \end{cases}
    %     \end{equation*}
    %     本征函数集由对应齐次方程和齐次边界条件构成, 写出形式解为
    %     $$ w(x, t) = \sum_{n\geq0}g_n(t)\sin\frac{(2n+1)\pi}{2l}x$$
    %     其中$g_n(t)$应该满足
    %     $$ g_n(t) = \frac{2}{l}\int_0^lw(x, t)\sin\frac{(2n+1)\pi}{2l}xdx$$
    %     那么容易有
    %     \begin{gather*}
    %         g_n(0) = \frac{2}{l}\int_0^l\left(\cos\frac{\pi}{l}x - 1\right)\sin\frac{(2n+1)\pi}{2l}xdx= \frac{16}{\pi(2n+1)(4n^2+4n-3)}\\
    %         g_n^{'}(0) = \frac{2}{l}\int_0^l\left(\sin\frac{\pi}{2l}x\right)\sin\frac{(2n+1)\pi}{2l}xdx = \delta_{0, n}
    %     \end{gather*}
    %     代入方程得到
    %     $$ \sum_{n\geq0}\left[g^{''}_n(t) + \left(\frac{(2n+1)a\pi}{2l}\right)^2g_n(t)\right]\sin\frac{(2n+1)\pi}{2l}x = \left(\frac{a\pi}{l}\right)^2\cos\frac{a\pi}{l}t$$
    %     将上式右端也按本征函数展开
    %     $$\left(\frac{a\pi}{l}\right)^2\cos\frac{a\pi}{l}t = \sum_{n\geq0}B_n\sin\frac{(2n+1)\pi}{2l}x $$
    %     其中
    %     $$ B_n(t) = \left(\frac{a\pi}{l}\right)^2\cos\frac{a\pi}{l}t\cdot\frac{2}{l}\int_0^l\sin\frac{(2n+1)\pi}{2l}xdx = \frac{4}{(2n+1)\pi}\left(\frac{a\pi}{l}\right)^2\cos\frac{a\pi}{l}t$$
    %     得到关于$g_n(t)$的非齐次常微分方程
    %     $$ g_n^{''}(t) + \left(\frac{(2n+1)a\pi}{2l}\right)^2g_n(t) =  B_n(t)$$
    %     解得
    %     $$ g_n(t) = C_n\cos\left(\frac{(2n+1)a\pi}{2l}\right)t + D_n\sin\left(\frac{(2n+1)a\pi}{2l}\right)t + \frac{16}{(2n-1)(2n+1)(2n+3)\pi}\cos\frac{a\pi}{l}t$$
    %     代入
    %     $$ g_n(0) = C_n + \frac{16}{(2n-1)(2n+1)(2n+3)\pi} = \frac{16}{\pi(2n+1)(4n^2+4n-3)}$$
    %     解得$C_n = 0$
    %     则
    %     $$ g^{'}(0) = \frac{(2n+1)a\pi}{2l}D_n = \delta_{0, n}$$
    %     解得
    %     \begin{gather*}
    %         D_0 = \frac{2l}{a\pi}\\
    %         D_n = 0, \quad n=1, 2, \cdots
    %     \end{gather*}
    %     所以
    %     $$ w(x, t) = \frac{2l}{a\pi}\sin{\frac{a\pi}{2l}t}\sin{\frac{\pi}{2l}x} + \sum_{n\geq1}$$
    % \end{itemize}
    % \end{solution}

    \begin{problem}{2}
        求解下列定界问题
        \begin{equation*}
            \begin{cases}
                    \frac{\partial u}{\partial t} = \kappa \frac{\partial^2 u}{\partial x^2},\quad 0 < x < l, t > 0\\
                    u(0, t) = Ae^{-\alpha^2\kappa t}, \quad u(l, t) = Be^{-\beta^2\kappa t}\\
                    u(x, 0) = 0
            \end{cases}
        \end{equation*}
    \end{problem}
    \begin{solution}
    % 记$$ \phi(t) = Ae^{-\alpha^2\kappa t}, \quad \psi(t) = Be^{-\beta^2\kappa t}$$
    % 做辅助函数
    % $$ v(x, t) = \left(1-\frac{x}{l}\right)\phi(t) + \frac{x}{l}\psi(t)$$
    % 则容易验证$w(x, t) = u(x, t) - v(x, t)$满足如下定解问题
    % \begin{equation*}
    %     \begin{cases}
    %             \frac{\partial w}{\partial t} = \kappa \frac{\partial^2 w}{\partial x^2} - \frac{\partial v}{\partial t}\\
    %             w(0, t) = w(l, t) = 0\\
    %             w(x, 0) = -v(x, 0)
    %     \end{cases}
    % \end{equation*}
    % 利用本征函数法, 假设解有如下形式
    % $$ w(x, t) + $$
    设辅助函数有如下形式
    $$ v(x, t) = f(x)e^{-\alpha^2\kappa t} + g(x)e^{-\beta^2\kappa t}$$
    我们希望$w(x, t) = u(x, t) - v(x, t)$同时满足齐次方程和齐次边界条件, 则
    \begin{equation*}
        \begin{aligned}
            \frac{\partial w}{\partial t} &= \frac{\partial u}{\partial t} - \frac{\partial v}{\partial t}\\
            &= \kappa \frac{\partial^2 u}{\partial x^2} -  \frac{\partial v}{\partial t}\\
            &= \kappa \left[\frac{\partial^2 w}{\partial x^2} + \frac{\partial^2 v}{\partial x^2}\right] -  \frac{\partial v}{\partial t}\\
            &= \kappa \frac{\partial^2 w}{\partial x^2} + \left(f^{''}(x) + \alpha^2f(x)\right)\kappa e^{-\alpha^2\kappa t} + \left(g^{''}(x) + \beta^2g(x)\right)\kappa e^{-\beta^2\kappa t}
        \end{aligned}
    \end{equation*}
    令
    \begin{gather*}
        f^{''}(x) + \alpha^2f(x) = 0\\
        g^{''}(x) + \beta^2g(x) = 0
    \end{gather*}
    再代入边界条件
    \begin{gather*}
        w(0, t) = Ae^{-\alpha^2\kappa t} - f(0)e^{-\alpha^2\kappa t} + g(0)e^{-\beta^2\kappa t}\\
        w(l, t) = Be^{-\beta^2\kappa t} - f(l)e^{-\alpha^2\kappa t} + g(l)e^{-\beta^2\kappa t}
    \end{gather*}
    令
    \begin{gather*}
        f(0) = A, \quad f(l) = 0\\
        g(0) = 0, \quad g(l) = B
    \end{gather*}
    解得
    $$ f(x) = -A\frac{\sin(\alpha(x-l))}{\sin(\alpha l)}, \quad g(x) = B\frac{\sin(\beta x)}{\sin(\beta l)}$$
    且$w$满足方程
    \begin{equation*}
        \begin{cases}
                \frac{\partial w}{\partial t} = \kappa \frac{\partial^2 w}{\partial x^2}\\
                u(0, t) = u(l, t) = 0\\
                u(x, 0) = -f(x) - g(x)
        \end{cases}
    \end{equation*}
    假设解有如下形式
    $$ w(x, t) = \sum_{n\geq1}g_n(t)\sin{\frac{n\pi}{l}x}$$
    代入方程得到关于$g_n(t)$的方程
    $$ g_n^{'}(t) + \kappa \left(\frac{n\pi}{l}\right)^2g_n(t) = 0$$
    解得
    $$ g_n(t) = C_nexp\left\{-\kappa \left(\frac{n\pi}{l}\right)^2t\right\}$$
    利用初始条件
    $$ w(x, 0) = \sum_{n\geq1}C_n\sin{\frac{n\pi}{l}x} = -f(x) - g(x)$$
    可以确定常数$C_n$
    代回即可得原方程的解
    $$ u(x, t) = -A\frac{\sin(\alpha(x-l))}{\sin(\alpha l)}e^{-\alpha^2\kappa t} +  B\frac{\sin(\beta x)}{\sin(\beta l)}e^{-\beta^2\kappa t} + \sum_{n\geq1}C_nexp\left\{-\kappa \left(\frac{n\pi}{l}\right)^2t\right\}\sin{\frac{n\pi}{l}x}$$
    \end{solution}
\end{document}